\documentclass[12pt]{article}
 \usepackage{caption}
\usepackage{subcaption}
\include{latex_env}

\rhead{Worksheet Five}

\begin{document}

{\bf GROUP WORKSHEET - TO BE HANDED IN}

%{\bf INDIVIDUAL WORKSHEETS - not for handing in}
{\bf Names} (First only, no student numbers):\\
\vspace{0.5in}

Before you begin you should have read and worked through Lab 5.

### Problem Tolerances
The Runge-Kutta algorithm with adaptive time steps will, in general, be more efficient and accurate than same
algorithm with fixed time steps. In other words, greater accuracy can usually be achieved in fewer time steps. For the given set of Daisyworld parameters and initial conditions:

1.  Decrease the error tolerances and compare the plots. You will note that as the error tolerances are decreased, the plots approach the one created by the algorithm with fixed time steps. What does this imply?

2.  Compare the Daisyworld plot to a plot of the stepsizes. Do you see a correlation between stepsize and the shape of the curve?

3.  Play with the tolerances and see if you can re-create (roughly) the same plot but in fewer time steps.

\end{document}