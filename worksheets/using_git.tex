\documentclass[11pt]{article}
\usepackage{geometry}
\geometry{letterpaper,top=50pt,hmargin={20mm,20mm},headheight=15pt} 

\usepackage{graphicx,amsmath,fancyhdr,times,setspace}
\usepackage{upquote,framed,cancel}
\usepackage{pdfpages}
\usepackage{url}
\usepackage[colorlinks=true, urlcolor=blue, linkcolor=blue, citecolor=blue, pdfborder={0 0 0}]{hyperref}

\usepackage{caption}

\oddsidemargin -0in %left margin on odd-numbered pages is 1in + 0in
\topmargin -.5in %top margin is 1in -0.5in
\headheight 15pt

\textwidth 6.5in %width of text
\textheight 9in % size of page
\setlength{\parindent}{0.0in}
\setlength{\parskip}{10pt}
\renewcommand{\thefootnote}{\fnsymbol{footnote}}

% Susan's definitions

\newcommand{\Mathat}{\;{\rm at}\;}
\newcommand{\Mathand}{\;{\rm and}\;}

\newcommand{\pa}{\partial}

\newcommand{\dc}{\pa c}
\newcommand{\de} {\pa \eta}
\newcommand{\deh} {\pa h}
\newcommand{\dphi} {\pa \phi}
\newcommand{\dpres} {\pa p}
\newcommand{\dpsi} {\pa \psi}
\newcommand{\dr}{\pa r}
%\newcommand{\dr}{\pa \rho}
\newcommand{\drho}{\pa \rho}
\newcommand{\ds}{\pa s}
\newcommand{\dS}{\pa S}
\newcommand{\dt}{\pa t}
\newcommand{\dT}{\pa T}
\newcommand{\dTheta}{\pa \Theta}
\newcommand{\du}{\pa u}
\newcommand{\dU}{\pa U}
\newcommand{\dv}{\pa v}
\newcommand{\dV}{\pa V}
\newcommand{\dw}{\pa w}
\newcommand{\dx}{\pa x}
\newcommand{\dxi}{\pa \xi}
\newcommand{\dy}{\pa y}
\newcommand{\dz}{\pa z}
\newcommand{\dzeta}{\pa \zeta}

\newcommand {\ddc} {\frac {\pa}{\dc}}
\newcommand {\ddp} {\frac {\pa}{\dpres}}
\newcommand {\ddr} {\frac {\pa}{\dr}}
\newcommand {\dds} {\frac {\pa}{\ds}}
\newcommand {\ddt} {\frac {\pa}{\dt}}
\newcommand {\DDt} {\frac {D}{Dt}}
\newcommand {\ddx} {\frac {\pa}{\dx}}
\newcommand {\ddy} {\frac {\pa}{\dy}}
\newcommand {\ddz} {\frac {\pa}{\dz}}
\newcommand {\dedt} {\frac {\de}{\dt}}
\newcommand {\dedx} {\frac {\de}{\dx}}
\newcommand {\dedy} {\frac {\de}{\dy}}
\newcommand {\dhdc} {\frac {\deh}{\dc}}
\newcommand {\dhdr} {\frac {\deh}{\dr}}
\newcommand {\dhdt} {\frac {\deh}{\dt}}
\newcommand {\dhdx} {\frac {\deh}{\dx}}
\newcommand {\dhdy} {\frac {\deh}{\dy}}
\newcommand {\dpdc}{\frac {\pa p}{\dc}}
\newcommand {\dpdr}{\frac {\pa p}{\dr}}
\newcommand {\dpds}{\frac {\pa p}{\ds}}
\newcommand {\dpdt}{\frac {\pa p}{\dt}}
\newcommand {\dpdx} {\frac {\pa p}{\dx}}
\newcommand {\dpdy} {\frac {\pa p}{\dy}}
\newcommand {\dpdz} {\frac {\pa p}{\dz}}
\newcommand {\dphidx} {\frac {\dphi}{\dx}}
\newcommand {\dphidy} {\frac {\dphi}{\dy}}
\newcommand {\dphidz} {\frac {\dphi}{\dz}}
\newcommand {\dpsidx} {\frac {\dpsi}{\dx}}
\newcommand {\dpsidy} {\frac {\dpsi}{\dy}}
\newcommand {\dpsidz} {\frac {\dpsi}{\dz}}
\newcommand {\drhodt} {\frac {\pa \rho}{\dt}}
\newcommand {\DrhoDt} {\frac {D \rho}{Dt}}
\newcommand {\drhodc} {\frac {\pa \rho}{\dc}}
\newcommand {\drhods} {\frac {\pa \rho}{\ds}}
\newcommand {\drhodx} {\frac {\pa \rho}{\dx}}
\newcommand {\drhody} {\frac {\pa \rho}{\dy}}
\newcommand {\drhodz} {\frac {\pa \rho}{\dz}}
\newcommand {\dSdt}{\frac {\dS}{\dt}}
\newcommand {\dsdz}{\frac {\pa s}{\dz}}
\newcommand {\dTds}{\frac {\dT}{\pa s}}
\newcommand {\dTdt}{\frac {\dT}{\dt}}
\newcommand {\DTDt}{\frac {DT}{Dt}}
\newcommand {\dTdx}{\frac {\dT}{\dx}}
\newcommand {\dTdy}{\frac {\dT}{\dy}}
\newcommand {\dTdz}{\frac {\pa T}{\dz}}
\newcommand {\dThetadt} {\frac {\dTheta}{\dt}}
\newcommand {\dThetadx} {\frac {\dTheta}{\dx}}
\newcommand {\dThetady} {\frac {\dTheta}{\dy}}
\newcommand {\dudt} {\frac {\du}{\dt}}
\newcommand {\dUdt} {\frac {\dU}{\dt}}
\newcommand {\DuDt} {\frac {Du}{Dt}}
\newcommand {\dudc} {\frac {\du}{\dc}}
\newcommand {\duds} {\frac {\du}{\ds}}
\newcommand {\dudx} {\frac {\du}{\dx}}
\newcommand {\dUdx} {\frac {\dU}{\dx}}
\newcommand {\dudy} {\frac {\du}{\dy}}
\newcommand {\dUdy} {\frac {\dU}{\dy}}
\newcommand {\dudz} {\frac {\du}{\dz}}
\newcommand {\dvdt} {\frac {\dv}{\dt}}
\newcommand {\dVdt} {\frac {\dV}{\dt}}
\newcommand {\DvDt} {\frac {Dv}{Dt}}
\newcommand {\dvdc} {\frac {\dv}{\dc}}
\newcommand {\dvdr} {\frac {\dv}{\dr}}
\newcommand {\dvds} {\frac {\dv}{\ds}}
\newcommand {\dvdx} {\frac {\dv}{\dx}}
\newcommand {\dVdx} {\frac {\dV}{\dx}}
\newcommand {\dvdy} {\frac {\dv}{\dy}}
\newcommand {\dVdy} {\frac {\dV}{\dy}}
\newcommand {\dvdz} {\frac {\dv}{\dz}}
\newcommand {\dwdt} {\frac {\dw}{\dt}}
\newcommand {\DwDt} {\frac {Dw}{Dt}}
\newcommand {\dwdx} {\frac {\dw}{\dx}}
\newcommand {\dwdy} {\frac {\dw}{\dy}}
\newcommand {\dwdz} {\frac {\dw}{\dz}}
\newcommand {\dydt} {\frac {\dy}{\dt}}
\newcommand {\dydx} {\frac {\dy}{\dx}}
\newcommand {\dzetadt} {\frac {\pa \zeta}{\dt}}
\newcommand {\dzetadx} {\frac {\pa \zeta}{\dx}}
\newcommand {\dzetady} {\frac {\pa \zeta}{\dy}}
\newcommand {\dzetadz} {\frac {\pa \zeta}{\dz}}
\newcommand {\dzds}{\frac {\dz}{\ds}}
\newcommand {\dzdt} {\frac {\dxi}{\dt}}
\newcommand {\dzdx} {\frac {\dxi}{\dx}}
\newcommand {\dzdy} {\frac {\dxi}{\dy}}


\newcommand{\gp}{g^\prime}
\newcommand{\om}{\omega}
\newcommand{\sgn}{{\rm sgn}}
\newcommand{\sech}{{\rm sech}}

\renewcommand{\deg}[1]{#1^\circ}
\newcommand{\textfrac}[2]{\textstyle \frac {#1} {#2}}
\newcommand{\dotprod}{\cdot}
\newcommand{\bigo}{{\cal O}}
\newcommand{\Kappa}{{\cal K}}
\newcommand{\del}{{\nabla}}
\newcommand{\grad}{{\nabla}}
\newcommand{\cross}{{\times}}
\renewcommand{\div}{{\nabla \cdot}}
\newcommand{\ul}{\underline}
\newcommand{\Chi}{{\cal X}}

\font\sm = cmex10
\font\mbigsm = cmex10 scaled \magstep3
\font\bigsm = cmex10 scaled \magstep4
\font\vbigsm = cmex10 scaled \magstep5

\def\bigsum#1{\hbox{$\textfont3=\vbigsm \displaystyle\sum\limits#1$}}
\def\bigint#1{\hbox{$\textfont3=\bigsm \displaystyle\int#1$}}
\def\medsum#1{\hbox{$\textfont3=\mbigsm \displaystyle\sum\limits#1$}}

\newcommand{\ppt}{$^\circ\!/\!_\circ\!_\circ$}



% Some new stuff

\newtheorem{question}{Question}

\newcounter{lnum}
\newenvironment{abbrevlist}%
  {\begin{list}{$\bullet$}{\setlength{\leftmargin}{2em}%
               \setlength{\itemindent}{0em}%
               \setlength{\itemsep}{0pt}%
               \setlength{\parsep}{0pt}%
               \setlength{\topsep}{2pt}%
               \usecounter{lnum} } }{\end{list}}

\newenvironment{abbrevnum}%
{\begin{list}{\arabic{lnum}.}{\setlength{\leftmargin}{2em}%
			\setlength{\itemindent}{0em}%
			\setlength{\itemsep}{0pt}%
			\setlength{\parsep}{0pt}%
			\setlength{\topsep}{2pt}%
			\usecounter{lnum} } }{\end{list}}


\pagestyle{fancy}
\lhead{}
\chead{ }
\fancyfoot[C]{\tiny Worksheets and assignments for Numeric Course by Susan Allen,
  Rachel White, Phil Austin and The University of British Columbia are licensed under a Creative Commons Attribution 3.0 Unported License}

\setcounter{tocdepth}{3}

\begin{document}

\section*{Using Git for your Project}

Using git or another type of version control is highly recommended for collaboratively coding your project.

It is also recommended that you write your code mostly in functions put in modules (.py files).  You can call these functions from a simple notebook.  The reason for this design is that .py files are pretty straightforward to merge whereas .ipynb files are messy to merge.

\subsection*{Set-up}

\begin{enumerate}
\item The code-keeper sets up a ``central'' repo on github.
\item Decision Point: direct commits by team or pull requests.
\begin{enumerate}
\item Direct Commits: each member clones the central repo
\item Pull Requests: each member forks the central repo and clones their fork
\end{enumerate}
\end{enumerate}

\subsection*{Working Session}

Below is how I work: its a suggestion. As I sit down to do some work on my piece, say the plotting subroutine
\begin{enumerate}
\item \texttt{git status}  First thing I check that I left everything nicely committed.  As I do the status command I am looking in particular for any modified files. These need to be committed.  You may also see untracked files: should some of those be committed?  Also note if you have commits that have not been pushed.  If so, you have made changes that your collaborators don't have yet.  Run your code to check where you were.  Does it work or do you have remaining bugs?  (Just note these).
\item Second thing: pull in my collaborators work.
\begin{enumerate}
\item Direct Commits: \texttt{git pull}
\item Pull Requests: On github, fetch from upstream, then on your computer \texttt{git pull}
\end{enumerate}
\item \texttt{git log} Read through the changes your teammates have made.
\item If you had no outstanding pushes, your code will now be identical to the ``central'' repo. If you have outstanding pushes, it will be the ``central'' repo + your changes. Run the code to check it works or works as well as it did before you pulled in the changes.
\item Now do your work.
\item \texttt{git add} \\ \texttt{git commit} Commit often with clear commit messages.  You future self and your team will appreciate!
\item When you have working code you can push your code to the
  ``central'' repo.  Do not push code that does not work!  For
  example, if someone else is working on the stability problem, they
  will not appreciate no plots!  Better plots they would like but no
  plots will make it hard for them to work.
\item Check for/pull in any new work by your collaborators
\begin{enumerate}
\item Direct Commits: \texttt{git pull}
\item Pull Requests: On github fetch from upstream \texttt{git pull}  and then on github, send pull request
\end{enumerate}
\item Push your work
\begin{enumerate}
\item Direct Commits: \texttt{git push}
\item Pull Requests: \texttt{git push}  and then on github, send pull request
\end{enumerate}
\item Do some more work
\item \texttt{git add} \\ \texttt{git commit}  Try very very hard not to leave uncommitted code.  Commit before you finish your work session.
\end{enumerate}

If you are doing direct commits, working as the code keeper is just like above.  If you are doing pull requests, the code keeper will be like direct commits.  The code keeper will also need to regularly review and pull in the pull requests from team mates.

My favorite git cheat sheet:\\
\url{https://www.atlassian.com/git/tutorials/atlassian-git-cheatsheet}

\end{document}


